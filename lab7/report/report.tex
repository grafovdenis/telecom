% !TeX spellcheck = ru_RU
\documentclass[a4paper,12pt]{extarticle}
\usepackage[utf8x]{inputenc}
\usepackage[T1,T2A]{fontenc}
\usepackage[russian]{babel}
\usepackage{hyperref}
\usepackage{indentfirst}
\usepackage{listings}
\usepackage{color}
\usepackage{here}
\usepackage{array}
\usepackage{multirow}
\usepackage{graphicx}

\usepackage{caption}
\renewcommand{\lstlistingname}{Программа} % заголовок листингов кода

\bibliographystyle{ugost2008ls}

\usepackage{listings}
\lstset{ %
extendedchars=\true,
keepspaces=true,
language=C,						% choose the language of the code
basicstyle=\footnotesize,		% the size of the fonts that are used for the code
numbers=left,					% where to put the line-numbers
numberstyle=\footnotesize,		% the size of the fonts that are used for the line-numbers
stepnumber=1,					% the step between two line-numbers. If it is 1 each line will be numbered
numbersep=5pt,					% how far the line-numbers are from the code
backgroundcolor=\color{white},	% choose the background color. You must add \usepackage{color}
showspaces=false				% show spaces adding particular underscores
showstringspaces=false,			% underline spaces within strings
showtabs=false,					% show tabs within strings adding particular underscores
frame=single,           		% adds a frame around the code
tabsize=2,						% sets default tabsize to 2 spaces
captionpos=t,					% sets the caption-position to top
breaklines=true,				% sets automatic line breaking
breakatwhitespace=false,		% sets if automatic breaks should only happen at whitespace
escapeinside={\%*}{*)},			% if you want to add a comment within your code
postbreak=\raisebox{0ex}[0ex][0ex]{\ensuremath{\color{red}\hookrightarrow\space}},
texcl=true,
inputpath=listings,                     % директория с листингами
}

\usepackage[left=2cm,right=2cm,
top=2cm,bottom=2cm,bindingoffset=0cm]{geometry}

%% Нумерация картинок по секциям
\usepackage{chngcntr}
\counterwithin{figure}{section}
\counterwithin{table}{section}

%%Точки нумерации заголовков
\usepackage{titlesec}
\titlelabel{\thetitle.\quad}
\usepackage[dotinlabels]{titletoc}

%% Оформления подписи рисунка
\addto\captionsrussian{\renewcommand{\figurename}{Рисунок}}
\captionsetup[figure]{labelsep = period}

%% Подпись таблицы
\DeclareCaptionFormat{hfillstart}{\hfill#1#2#3\par}
\captionsetup[table]{format=hfillstart,labelsep=newline,justification=centering,skip=-10pt,textfont=bf}

%% Путь к каталогу с рисунками
\graphicspath{{fig/}}

\usepackage{minted}

\begin{document}	% начало документа

% Титульная страница
\begin{titlepage}	% начало титульной страницы

	\begin{center}		% выравнивание по центру

		\large Санкт-Петербургский политехнический университет Петра Великого\\
		\large Институт компьютерных наук и технологий \\
		\large Кафедра компьютерных систем и программных технологий\\[6cm]
		% название института, затем отступ 6см
		
		\huge Телекоммуникационные технологии\\[0.5cm] % название работы, затем отступ 0,5см
		\large Отчет по лабораторной работе №7\\[0.1cm]
		\large Помехоустойчивое кодирование\\[5cm]

	\end{center}


	\begin{flushright} % выравнивание по правому краю
		\begin{minipage}{0.25\textwidth} % врезка в половину ширины текста
			\begin{flushleft} % выровнять её содержимое по левому краю

				\large\textbf{Работу выполнил:}\\
				\large Графов Д.И.\\
				\large {Группа:} 33531/2\\
				
				\large \textbf{Преподаватель:}\\
				\large Богач Н.В.

			\end{flushleft}
		\end{minipage}
	\end{flushright}
	
	\vfill % заполнить всё доступное ниже пространство

	\begin{center}
	\large Санкт-Петербург\\
	\large \the\year % вывести дату
	\end{center} % закончить выравнивание по центру

\thispagestyle{empty} % не нумеровать страницу
\end{titlepage} % конец титульной страницы

\vfill % заполнить всё доступное ниже пространство


% Содержание
\setcounter{page}{2}
% Содержание
\renewcommand\contentsname{\centerline{Содержание}}
\tableofcontents
\newpage



\section{Цель работы}
\begin{itemize}
	\item Изучить методы помехоустойчивого кодирования и сравнить их свойства.
\end{itemize}
\section{Задачи работы}
\begin{itemize}
	\item Провести кодирование/декодирование сигнала, полученного с помощью функции randerr кодом Хэмминга 2-мя способами: с помощью встроенных функций encode/decode, а также через создание проверочной и генераторной матриц и вычисление синдрома. Оценить корректирующую способность кода.
	\item Выполнить кодирование/декодирование циклическим кодом, кодом БЧХ. Оценить корректирующую способность кода.
\end{itemize}

\section{Теоретическая информация}
Помехоустойчивое кодирование — кодирование, предназначенное для передачи данных по каналам с помехами, обеспечивающее исправление возможных ошибок передачи вследствие помех. Для обнаружения ошибок используют коды обнаружения ошибок, для исправления — помехоустойчивые коды.

Помехоустойчивое кодирование предполагает введение в передаваемое сообщение, наряду с информационными, так называемых проверочных разрядов, формируемых в устройствах защиты от ошибок (кодерах-на передающем конце, декодерах — на приемном). Избыточность позволяет отличить разрешенную и запрещенную (искаженную за счет ошибок) комбинации при приеме, иначе одна разрешенная комбинация переходила бы в другую.

Помехоустойчивый код характеризуется тройкой чисел (n, k, d0), где n— общее число разрядов в передаваемом сообщении, включая проверочные (г), k=n-r - число информационных разрядов, d0— минимальное кодовое расстояние между разрешенными кодовыми комбинациями, определяемое как минимальное число различающихся бит в этих комбинациях. Число обнаруживаемых (tj и (или) исправляемых (t ) ошибок (разрядов) связано с параметром d0 соотношениями.

Коды Хемминга — простейшие линейные коды с минимальным расстоянием 3, то есть способные исправить одну ошибку. Код Хемминга может быть представлен в таком виде, что синдром:

$ \vec{s} = \vec{r}H^T $, где $\vec{r}$ — принятый вектор, будет равен номеру позиции, в которой произошла ошибка. Это свойство позволяет сделать декодирование очень простым.

Коды Боуза — Чоудхури — Хоквингема (БЧХ) являются подклассом циклических кодов. Их отличительное свойство — возможность построения кода БЧХ с минимальным расстоянием не меньше заданного.
\newpage

\newpage
\section{Ход выполнения работы}
Данная работа выполнялась на языке Python.

К сожалению, мной не были найдены какие-либо библиотечные решения помехоустойчивого кодирования, поэтому вся работа была проделана на "чистом" Python, за исключением матречных операций с использованием NumPy.

За основу работы был взят материал, полученный на лекции и лабораторном занятии.

С полным кодом можно ознакомиться по адресу:

\href{https://github.com/grafovdenis/telecom/tree/master/lab7/src/}{https://github.com/grafovdenis/telecom/tree/master/lab7/src/}

\section{Результаты работы}
\subsection{Код Хэмминга}

Рассмотрим функцию encode, реализующую кодирование 4-битного кода в код Хэмминга (7, 4).

\large {Листинг 1. encode}
\inputminted[
firstline=6,
lastline=16,
frame=lines,
framesep=2mm,
baselinestretch=1.2,
fontsize=\footnotesize,
linenos
]{python}{../src/noise_immunity_coding.py}

\large {Листинг 2. Результат работы encode}

\begin{lstlisting}
  Source | Encoded
[0 1 0 0] [0, 1, 0, 1, 0, 1, 0]
[0 0 1 1] [0, 1, 1, 1, 1, 0, 0]
[1 1 0 0] [1, 0, 0, 0, 0, 1, 1]
[0 0 1 1] [0, 1, 1, 1, 1, 0, 0]
[1 1 0 1] [0, 1, 1, 0, 0, 1, 1]
[0 0 1 1] [0, 1, 1, 1, 1, 0, 0]
[0 1 1 0] [1, 1, 0, 0, 1, 1, 0]
[1 0 0 1] [0, 0, 1, 1, 0, 0, 1]
[0 0 0 0] [0, 0, 0, 0, 0, 0, 0]
[1 1 1 1] [1, 1, 1, 1, 1, 1, 1]
[0 0 1 0] [1, 0, 0, 1, 1, 0, 0]
[0 0 1 0] [1, 0, 0, 1, 1, 0, 0]
[0 0 0 1] [1, 1, 1, 0, 0, 0, 0]
[1 0 1 1] [1, 0, 1, 0, 1, 0, 1]
[1 1 0 0] [1, 0, 0, 0, 0, 1, 1]
[0 0 0 1] [1, 1, 1, 0, 0, 0, 0]
\end{lstlisting}

Рассмотрим кодирование конкретного примера 1011 и вычисление синдрома.

\large {Листинг 3. Вспомогательные функции для вычисления синдрома, нахождения и исправления ошибки}
\inputminted[
firstline=19,
lastline=37,
frame=lines,
framesep=2mm,
baselinestretch=1.2,
fontsize=\footnotesize,
linenos
]{python}{../src/noise_immunity_coding.py}

\large {Листинг 4. Запуск примера}
\inputminted[
firstline=52,
lastline=63,
frame=lines,
framesep=2mm,
baselinestretch=1.2,
fontsize=\footnotesize,
linenos
]{python}{../src/noise_immunity_coding.py}

\large {Листинг 5. Результат работы}
\begin{lstlisting}
Syndrome for correct message: 0
Syndrome example for: [1 0 1 1]
[1, 0, 1, 0, 1, 0, 1]
Let error be in 6 digit:
[1, 0, 1, 0, 1, 0, 0]
Error in 6 digit
Recovered message:
[1, 0, 1, 0, 1, 0, 1]
\end{lstlisting}

\newpage
Далее рассмотрим пример с пораждающей матрицей.

\large {Листинг 6. Пример с пораждающий матрицей}
\inputminted[
firstline=65,
frame=lines,
framesep=2mm,
baselinestretch=1.2,
fontsize=\footnotesize,
linenos
]{python}{../src/noise_immunity_coding.py}

\large {Листинг 7. Результат работы}
\begin{lstlisting}
Generator matrix example:
[[0 1 0 1]]
Encoded matrix: [[0 1 0 1 0 1 0]]
Damaged matrix: [[0 1 0 1 0 1 1]]
Result matrix : [[0 0 1 0 0 1 1]]
Damaged result: [[0 1 0 1 0 1 1]]
Syndrome of correct message: 0
Syndrome of damaged message: 4 (means that mistake is in 3rd digit)
\end{lstlisting}

\newpage
\subsection{Циклический код}
Рассмотрим пример работы с циклическим кодом для числа 1010.

\large {Листинг 8. cyclical.py}
\inputminted[
frame=lines,
framesep=2mm,
baselinestretch=1.2,
fontsize=\footnotesize,
linenos
]{python}{../src/cyclical.py}

\newpage
\large {Листинг 9. Результат работы}
\begin{lstlisting}
Cyclical code for: [1 0 1 0]
[[1 0 0 1 1 1 0]]
Syndrome for correct message  : [[0 0 0]]
Syndrome for incorrect message: [[0 0 1]]
\end{lstlisting}

\newpage
\section{Выводы}
В ходе данной работы на языке Python были написаны функции для кодирования в код Хэмминга (7, 4) и вычисления синдома, был рассмотрен пример с пораждающей матрицей, а также с циклическим кодом.
 
Таким образом мной было рассмотрено помехоустойчивое кодирование
сигнала с помощью различных методов.

\end{document}

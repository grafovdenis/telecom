% !TeX spellcheck = ru_RU
\include{settings}
\usepackage{minted}

\begin{document}	% начало документа

% Титульная страница
\include{titlepage}

% Содержание
\setcounter{page}{2}
\include{ToC}


\section{Цель работы}
Изучение амплитудной модуляции/демодуляции сигнала.

\section{Программа работы}
\begin{enumerate}
	\item Сгенерировать однотональный сигнал низкой частоты.
	\item Выполнить амплитудную модуляцию (АМ) сигнала по закону
	$ u(t) = (1 + MU_m\cos\Omega t)\cos(\omega_0 t + \phi_0) $
	для различных значений глубины модуляции.
	\item Получить спектр модулированного сигнала.
	\item Выполнить модуляцию с подавлением несущей
	$ u(t) = MU_m\cos(\Omega t)\cos(\omega_0 t + \phi_0). $
	Получить спектр.
	\item Выполнить однополосную модуляцию:
	$ u(t) = U_m\cos(\omega t)\cos(\omega_0 t + \phi_0) + \frac{U_m}{2}\sum_{n=1}^{N}(M_n\cos(\omega_0 + \Omega_n) t + \phi_0 + \Phi_n)$ положив n=1.
	\item Выполнить синхронное детектирование и получить исходный
	однополосный сигнал.
	\item Рассчитать КПД модуляции.
	$ \eta_{AM} = \frac{U_m^2 M^2 / 4}{P_U} = \frac{M^2}{M^2 + 2}$
\end{enumerate}

\section{Теоретическая информация}
Модуляция (лат. modulatio — размеренность, ритмичность) — процесс изменения одного или нескольких параметров модулируемого несущего сигнала при помощи модулирующего сигнала.

Амплитудная модуляция — вид модуляции, при которой изменяемым параметром несущего сигнала является его амплитуда.

Разновидности амплитудной модуляции:
\begin{itemize}
	\item АМ с подавлением несущей (Suppressed carrier) - удаление несущего колебания.
	Демодуляция выполняется путём синхронного детектирования.
	\item Однополосная модуляция (Single sideband) - удаление одной из боковых полос.
\end{itemize}

Демодуляция АМ-сигнала может быть выполнена несколькими способами. Простейший путь - вычислить модуль входного АМ-сигнала, затем сгладить получившиеся однополярные косинусоидальные импульсы, пропуская их через фильтр низких частот.

\newpage
\section{Ход выполнения работы}
Данная работа выполнялась на языке Python.

Характеристики сигналов:
\begin{itemize}
	\item частота модулируемого сигнала: 10 Гц;
	\item амплитуда модулируемого сигнала: 1;
	\item частота несущей: 100 Гц;
	\item амплитуда несущей: 1.
\end{itemize}

Для начала, по заданным формулам был закодирован сигнал (строки 17-19).
Затем, была осуществлено детектирование закодированных сигналов: для первого примера с помощью функции filfit() от модуля закодированного сигнала; и для второго примера, с помощью умножения на несущую.

После этого, был произведён расчёт спектров полученных сигналов.

\large {Листинг 1. main.py}
\inputminted[
frame=lines,
framesep=2mm,
baselinestretch=1.2,
fontsize=\footnotesize,
linenos
]{python}{../src/amplitude_modulation.py}

\newpage
\section{Результаты работы}

\begin{figure}[H]
	\begin{center}
		\includegraphics[scale=0.7]{../out/Singleton_time.png}
		\caption{Амплитудная (однотональная) модуляция и демодуляция} 
		\label{pic:sine_time_0} % название для ссылок внутри кода
	\end{center}
\end{figure}

\begin{figure}[H]
	\begin{center}
		\includegraphics[scale=0.7]{../out/Singleton_frequency.png}
		\caption{Спектр промодулированного сигнала} 
		\label{pic:sine_freq_0} % название для ссылок внутри кода
	\end{center}
\end{figure}

\begin{figure}[H]
	\begin{center}
		\includegraphics[scale=0.7]{../out/Suppressed_time.png}
		\caption{Модуляция и демодуляция с подавлением несущей} 
		\label{pic:sine_time_1} % название для ссылок внутри кода
	\end{center}
\end{figure}

\begin{figure}[H]
	\begin{center}
		\includegraphics[scale=0.7]{../out/Suppressed_frequency.png}
		\caption{Спектр промодулированного сигнала} 
		\label{pic:sine_freq_1} % название для ссылок внутри кода
	\end{center}
\end{figure}

\begin{figure}[H]
	\begin{center}
		\includegraphics[scale=0.7]{../out/Single_time.png}
		\caption{Однополосная модуляция} 
		\label{pic:sine_time_2} % название для ссылок внутри кода
	\end{center}
\end{figure}

\begin{figure}[H]
	\begin{center}
		\includegraphics[scale=0.7]{../out/Single_frequency.png}
		\caption{Спектр промодулированного сигнала} 
		\label{pic:sine_freq_2} % название для ссылок внутри кода
	\end{center}
\end{figure}

\newpage
\section{Выводы}
В ходе выполнения работы я ознакомился с амплитудной модуляцией и демодуляцией, их разновидностями.
Для однотональной модуляции и модуляции с подавлением несущей были продемонстрированы частотные и временные характеристики закодированных и декодированных сигналов.
Декодировать последний сигнал не удалось.
\end{document}

% !TeX spellcheck = ru_RU
\include{settings}
\usepackage{minted}

\begin{document}	% начало документа

% Титульная страница
\include{titlepage}

% Содержание
\setcounter{page}{2}
\include{ToC}


\section{Цель работы}
Изучение частотной и фазовой модуляции/демодуляции сигнала.

\section{Программа работы}
\begin{enumerate}
	\item Сгенерировать однотональный сигнал низкой частоты.
	\item Выполнить фазовую модуляцию/демодуляцию сигнала по закону
	
	$ u(t) = U_m cos(\Omega t + k s(t)) $.
	\item Получить спектр модулированного сигнала.
	\item Выполнить частотную модуляцию/демодуляцию по закону 
	
	$ u(t) = U_m cos (\omega_0 t + k \int_{0}^{t} s(t) dt + \theta_0) $.
	
\end{enumerate}

\section{Теоретическая информация}
Модуляция (лат. modulatio — размеренность, ритмичность) — процесс изменения одного или нескольких параметров модулируемого несущего сигнала при помощи модулирующего сигнала.

Фазовая и частотная модуляция тесно связаны друг с другом, и благодаря чему получили общее название "угловая модуляция"(УМ; angle modulation).

\subsection{Фазовая модуляция}
Фазовая модуляция (ФМ, Phase Modulation) — вид модуляции, при которой фаза несущего колебания изменяется прямо пропорционально информационному сигналу. Фазомодулированный сигнал u(t) имеет следующий вид:

$ u(t) = U_m cos(\Omega t + k s(t)) $.

$ U_m $ — амплитуда сигнала; $ s(t) $ — модулирующий информационный сигнал; k – постоянная; $ \Omega $ — угловая частота несущего сигнала; t — время.

По характеристикам фазовая модуляция близка к частотной модуляции. В случае синусоидального модулирующего (информационного) сигнала, результаты частотной и фазовой модуляции совпадают.

\subsection{Частотная модуляция}
Частотная модуляция (ЧМ, Frequency Modulation) — вид аналоговой модуляции, при которой, частота несущей изменяется по закону модулирующего низкочастотного сигнала. Амплитуда при этом остается постоянной.

Наибольшее отклонение частоты от среднего значения, называется девиацией.
В идеальном варианте, девиация должна быть прямо пропорционально амплитуде модулирующего колебания.

\subsection{Демодуляция УМ}
Демодуляция УМ-сигнала может выполняться различными способами. Наиболее радикальных подход - вычислить аналитический сигнал и выделить его фазовую функцию. Далее для демодуляции из фазовой функции вычитается линейное слагаемое $ \omega_0 t $, соотвествующее немодулированной несущей.

\section{Ход выполнения работы}
Данная работа выполнялась на языке Python.

Характеристики сигналов:
\begin{itemize}
	\item частота модулируемого сигнала: 10 Гц;
	\item амплитуда модулируемого сигнала: 1;
	\item частота несущей: 50 Гц;
	\item амплитуда несущей: 1.
\end{itemize}

Для начала, по заданным формулам был закодирован сигнал (строки 16-23).

Затем был вычислен аналитический сигнал и его фазовая фукция. После чего была проведена демодуляция.

После этого был произведён расчёт спектров полученных сигналов.

\large {Листинг 1. main.py}
\inputminted[
frame=lines,
framesep=2mm,
baselinestretch=1.2,
fontsize=\footnotesize,
linenos
]{python}{../src/pm_fm.py}

\newpage
\section{Результаты работы}

\begin{figure}[H]
	\begin{center}
		\includegraphics[scale=0.7]{../out/PM_time.png}
		\caption{Фазовая модуляция и демодуляция} 
		\label{pic:sine_time_0} % название для ссылок внутри кода
	\end{center}
\end{figure}

\begin{figure}[H]
	\begin{center}
		\includegraphics[scale=0.7]{../out/PM_frequency.png}
		\caption{Спектр промодулированного сигнала} 
		\label{pic:sine_freq_0} % название для ссылок внутри кода
	\end{center}
\end{figure}

\begin{figure}[H]
	\begin{center}
		\includegraphics[scale=0.7]{../out/FM_time.png}
		\caption{Частотная модуляция и демодуляция} 
		\label{pic:sine_time_1} % название для ссылок внутри кода
	\end{center}
\end{figure}

\begin{figure}[H]
	\begin{center}
		\includegraphics[scale=0.7]{../out/FM_frequency.png}
		\caption{Спектр промодулированного сигнала} 
		\label{pic:sine_freq_1} % название для ссылок внутри кода
	\end{center}
\end{figure}

\newpage
\section{Выводы}
В ходе выполнения работы я ознакомился с угловой модуляцией и демодуляцией, их разновидностями.
Для фазовой и частотной модуляции были продемонстрированы частотные и временные характеристики закодированных и декодированных сигналов.
\end{document}

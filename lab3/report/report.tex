% !TeX spellcheck = ru_RU
\include{settings}
\usepackage{minted}

\begin{document}	% начало документа

% Титульная страница
\include{titlepage}

% Содержание
\setcounter{page}{2}
\include{ToC}


\section{Цель работы}
Изучить воздействие ФНЧ на тестовый сигнал с шумом.

\section{Программа работы}
\begin{itemize}
	\item Сгенерировать гармонический сигнал с шумом и синтезировать ФНЧ.
	\item Получить сигнал во временной и частотной областях до и после фильтрации.
	\item Сделать выводы о воздействии ФНЧ на спектр сигнала.
\end{itemize}

\section{Теоретическая информация}
Аддитивный белый гауссовский шум (АБГШ, англ. AWGN) — вид мешающего воздействия в канале передачи информации. Характеризуется равномерной, то есть одинаковой на всех частотах, спектральной плотностью мощности, нормально распределёнными временными значениями и аддитивным способом воздействия на сигнал. Наиболее распространённый вид шума, используемый для расчёта и моделирования систем радиосвязи. Термин «аддитивный» означает, что данный вид шума суммируется с полезным сигналом и статистически не зависим от сигнала. В противоположность аддитивному, можно указать мультипликативный шум — шум, перемножающийся с сигналом.

\newpage
\section{Ход выполнения работы}
Данная работа выполнялась на языке Python.

В качестве шума был выбран аддитивный белый гауссовский шум с параметрами: 
\begin{itemize}
	\item математическое ожидание: 0;
	\item среднеквадратичное отклонение: 0.5.
\end{itemize}
Затем данный шум был прибавлен к исходному гармоническому сигналу с амплитудой равной 1 и частотой равной 20 Гц.

Далее была осуществлена фильтрация c помощью filtfilt() пакета signal.
Затем были получены графики во временных и частотных областях исходного и отфильтрованного сигнала.

\large {Листинг 1. main.py}
\inputminted[
frame=lines,
framesep=2mm,
baselinestretch=1.2,
fontsize=\footnotesize,
linenos
]{python}{../src/main.py}

\newpage
\large {Результат работы}
\begin{figure}[H]
	\begin{center}
		\includegraphics[scale=0.7]{../out/sine_time.png}
		\caption{Синусоидальный сигнал} 
		\label{pic:sine_time} % название для ссылок внутри кода
	\end{center}
\end{figure}

\begin{figure}[H]
	\begin{center}
		\includegraphics[scale=0.7]{../out/sine_freq.png}
		\caption{Спектр синусоидального сигнала} 
		\label{pic:sine_freq} % название для ссылок внутри кода
	\end{center}
\end{figure}

\newpage
\section{Выводы}
В ходе выполнения работы мной была осуществлена фильтрация гармонического сигнала.

После фильтрации удалось сохранить значимую часть сигнала: была сохранена частота. Однако, в два раза уменьшилась амплитуда.

Спектрограмма показывает относительный успех фильтрации: шумов стало значительно меньше, особенно в области высоких частот.

Непостоянство амплитуды обусловлено остаточным низкочастотным шумом.
\end{document}
